\chapter*{Zusammenfassung}
%% ==============================

In dieser Arbeit wird das Spiel Latrunculi betrachtet und eine digitale Version für das Landesmuseum Birkenfeld umgesetzt. Zum Zeitpunkt dieser Arbeit begleiten uns Kontaktbeschränkungen im Alltag, daher war die Idee dem Museum eine Alternative bieten zu können, die auch online abgerufen werden kann. Daher wurde mit Hilfe der Unity Game Engine eine unter Windows lauffähige Version, sowie eine webbasierte implementiert. Hierzu wurden überlieferte Spielregeln umgesetzt und eigene Ideen für ein Spielende festgelegt, da historisch nicht alle Informationen überliefert wurden. Zur Umsetzung habe ich mich mit dem deterministischen MiniMax-Algorithmus auseinandergesetzt, sowie die Verbesserung Alpha-Beta betrachtet. Diese implementierte künstliche Intelligenz wurde genutzt um das Spielen gegen den Computer zu ermöglichen und zur Simulation bei der zwei KIs gegeneinander antreten. Das Verhalten beider Möglichkeiten wurde im 6. Kapitel analysiert. Bei dieser Analyse hat sich herausgestellt, dass es wenig sinnvoll ist KIs mit den selben Konfigurationen gegeneinander antreten zu lassen. Diese KIs tendieren dazu sich gegenseitig auszuweichen. Weiterhin wurde das Verhalten gegen menschliche Spieler beobachtet und verbessert. Eine primitive Version wurde zu Beginn sehr leicht geschlagen, konnte aber mit einigen Anpassungen in der Analyse der Situation verbessert werden.


%Hier steht eine Kurzzusammenfassung (Abstract) der Arbeit. Stellen Sie kurz und präzise Ziel und Gegenstand der Arbeit, die angewendeten Methoden, sowie die Ergebnisse der Arbeit dar. Halten Sie dabei die ersten Punkten eher kurz und fokussieren Sie die Ergebnisse. Bewerten Sie auch die Ergebnissen und ordnen Sie diese in den Kontext ein.

%Die Kurzzusammenfassung sollte maximal 1 Seite lang sein.
