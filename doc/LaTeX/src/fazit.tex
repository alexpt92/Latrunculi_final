%% zusammenf.tex
%% $Id: zusammenf.tex 61 2012-05-03 13:58:03Z bless $
%%

\chapter{Diskussion und Ausblick}
\label{ch:fazit}
%% ==============================
Latrunculi wurde in dieser Arbeit digital umgesetzt und ist auf dem GitHub Repository \url{https://github.com/alexpt92/Latrunculi_final} abrufbar. Dabei wurden verschieden Einstellungen ausprobiert um zu evaluieren wie zielführend sich die implementierte KI verhält. Es wurden aggressive, defensive und ausgeglichene Konfigurationen mit verschiedenen Suchtiefen evaluiert. Bei den Versuchen ist aufgefallen, dass vor allem die ausgeglichene Variante weniger sinnvolle Spielzüge umsetzt. Am schwierigsten gegen den Computer zu gewinnen, war es mit defensiven Einstellungen. Hier hat die KI sich schnell in unschlagbare Positionen begeben und diese nur verlassen, wenn es zu einer Eroberung geführt hat. Die aggressive Konfiguration hat dafür stellenweise Spielzüge durchgeführt in denen innerhalb von drei aufeinanderfolgenden Zügen Steine erobert wurden. Dadurch, dass es möglich war die KI einzukesseln und zu gewinnen, wurde diese Einstellung als mittlerer Schwierigkeitsgrad festgelegt und die defensive Strategie als schwer. Die ausgeglichene Konfiguration wurde als leicht klassifiziert. Außerdem ist aufgefallen, dass eine höhere Suchtiefe zu insgesamt zielführenden Spielzügen führt. In der Simulation wurden die KIs zuerst mit jeweils den selben Einstellungen evaluiert. Das Problem dabei ist, dass beide die selbe Strategie verfolgen und so keine zielführenden Spielzüge Zustande kommen. Daher wurde für eine KI die defensive und für die andere die aggressive Konfiguration gewählt. Hier wurde beobachtet, dass die implementierten Spielmechaniken deutlicher werden. Die aggressive Einstellung versucht zu Beginn noch Steine anzugreifen, verliert dabei stellenweise aber eigene. Die andere KI orientiert sich schnell an den Eckpositionen und nutzt das aggressive Verhalten zum eigenen Vorteil. Da die Simulation vor allem für Vorführungszwecke genutzt werden soll, scheint diese Konfiguration am sinnvollsten zu sein.\\
Zusammenfassend kann man festhalten, dass die Schwierigkeit in der Implementierung einer zielführenden KI in der Evaluierungsfunktion liegt. Diese muss möglichst präzise die Situation analysieren können um daraus gute Spielzüge zu evaluieren. Das Problem bei Latrunculi ist, dass wenige Spielmechaniken und Strategien überliefert wurden, daher gibt es einen großen Spielraum an möglichen Umsetzungen. Dadurch, dass es auch zu vorzeitigen Einigungen auf den Sieger gekommen sein soll, macht es es schwieriger eine zielführende Simulation umzusetzen.
Ein weiterer Punkt zur Verbesserung ist die Umsetzung des Alpha-Beta Algorithmus, der die Berechnung des MiniMax abkürzt und dadurch eine kürzere Laufzeit erreicht wird. Allerdings hat das wenig Auswirkungen auf die Effektivität der KI.

%%% Local Variables: 
%%% mode: latex
%%% TeX-master: "thesis"
%%% End: 
