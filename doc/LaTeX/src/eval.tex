%% eval.tex
%% $Id: eval.tex 61 2012-05-03 13:58:03Z bless $

\chapter{Evaluation}
\label{ch:Evaluierung}
%% ==============================
%Hier erfolgt der Nachweis, dass das in Kapitel~\ref{ch:Entwurf}
%entworfene Konzept funktioniert. 
%Leistungsmessungen einer Implementierung werden immer gerne gesehen.
Zur Evaluation wurde eine lauffähige Windows Version und Webversion dem Landesmuseum Birkenfeld zugesendet.

%% ==============================
\section{Launch beim Kunden}
%% ==============================
\label{ch:Evaluierung:sec:Launch}
Die erste spielbare Version mit den grundlegenden und erweiterten Features wurde dem Landesmuseum mittels GitHub und E-Mail zugesendet, sodass diese Version trotz anhaltender Kontaktbeschränkungen vor Ort getestet werden konnte. Herr Decker hat uns während einem GoToMeeting-Videoanruf die Testversion auf ihrem Gerät zeigen können. Dabei wurde klar, dass die grundlegenden und erweiterten Features funktionieren. Es wurde festgestellt, dass erfahrene Spieler der implementierten KI überlegen sind und regelmäßig gewinnen. Das Spiel gegen die KI wurde zuvor von Personen, die sich nicht mit Latrunculi auseinandergesetzt haben, getestet und verloren, allerdings auch nicht in jeder Runde. Das Hervorheben der erlaubten Positionen wurde auf dem vorhandenen Gerät nicht wieder entfernt, wenn der Spieler seine Züge zu schnell erledigt hat. Außerdem sind die Spielsteine teilweise zwischen den Zellen hängen geblieben und nicht auf der korrekten Position eingerastet. Diese beiden Punkte waren auf meinen Testgeräten nicht reproduzierbar, daher liegt die Vermutung nahe, dass es ein hardwarespezifisches Problem des genutzten Touch-Tisches vor Ort im Museum ist. Aufgrund der Kontaktbeschränkungen ist es aktuell auch nicht möglich, die Version persönlich vor Ort zu testen, um Fehlerquellen einzugrenzen. Ein weiterer Punkt, der aufgefallen war, ist, dass das Menü auf dem Gerät vor Ort zu klein wirkte und sich nicht an die Bildschirmgröße angepasst hat. Meine Geräte haben maximal eine Größe von 22 Zoll, sodass es zuvor nicht möglich war, es auf einem vergleichbaren Gerät mit einer Größe von 55 Zoll zu testen.
Ein weiterer Punkt, der sich gewünscht wurde, war, dass die KI deaktivierbar sein soll, damit das Spielen zwischen zwei Personen möglich ist.

%% ==============================
\section{Verhalten in der Simulation}
%% ==============================
\label{ch:Evaluierung:sec:Simulation}

\ldots

%% ==============================
\section{Zusammenfassung}
%% ==============================
\label{ch:Evaluierung:sec:zusammenfassung}

Am Ende sollten ggf. die wichtigsten Ergebnisse nochmal in \emph{einem} kurzen Absatz zusammengefasst werden.

%%% Local Variables: 
%%% mode: latex
%%% TeX-master: "thesis"
%%% End: 
