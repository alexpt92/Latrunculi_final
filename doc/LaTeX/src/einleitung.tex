%% Einleitung.tex
%% $Id: einleitung.tex 61 2012-05-03 13:58:03Z bless $
%%

\chapter{Einleitung}
\label{ch:Einleitung}
%% ==============================
Latrunculi soll historisch gesehen ein beliebtes und weit verbreitetes strategisches Brettspiel der Römer und Griechen gewesen sein. Die römischen Dichter Ovid und Martial haben dieses Spiel bereits erwähnt und beschrieben, dass neben den beiden Spielern, sogar Zuschauer bei sehr talentierten Spielern dazu gekommen sind. Die Spielsteine wurden als latrones(lat. für Soldat) bezeichnet. Da die Entstehung dieses Spiels so lange zurück liegt, wurden nicht alle Regeln überliefert, sodass verschieden Quellen hinzugezogen werden müssen um den Spielmechanismus zu klären. Das Latrunculi-Spielfeld soll aus einem Raster senkrechter und waagerechter Reihen bestanden haben. Weiterhin wurden wahrscheinlich die Spielsteine auf den Feldern und nicht auf den Linien platziert und bewegt. Dabei konnte keine fixe Größe des Spielbretts festgestellt werden, sodass verschieden große Spielfelder mit unterschiedlicher Anzahl an Feldern gefunden. Bei Ausgrabungen konnten Latrunculi-Bretter mit beispielsweise 7x7, 8x8, 9x10, 7x10 und 7x6 Feldern geborgen werden. Als Spielbretter dienten hierbei unter anderem Kalksteine oder Ziegelsteine, wie sie in Mainz oder in Hadrianswall in Großbritannien gefunden wurden. Betrachtet man Ovids Aussagen, kann man folgern, dass das Hauptziel darin bestand mit seinen Spielsteinen einen gegnerischen von zwei gegenüberliegenden Seiten zu einzufangen und somit aus dem Spiel zu nehmen. Weiterhin beschreibt er, dass es wichtig ist seine Steine paarweise zu bewegen, da dadurch verhindert wird, dass diese vom Gegner geschlagen werden können. Für diesen Angriffs- und Verteidigungsmechanismus konnten die Steine geradlinig vorwärts und rückwärts verschoben werden.

Die Einleitung besteht aus der Motivation, der Problemstellung, der Zielsetzung und einem erster Überblick über den Aufbau der Arbeit.

%% ==============================
\section{Motivation}
%% ==============================
\label{ch:Einleitung:sec:Motivation}

Warum ist das zu bearbeitende Themengebiet spannend und relevant?

%% ==============================
\section{Problemstellung}
%% ==============================
\label{ch:Einleitung:sec:Problemstellung}

Welches Problem/welche Probleme können in diesem Themengebiet identifiziert werden?

%% ==============================
\section{Zielsetzung}
%% ==============================
\label{ch:Einleitung:sec:Zielsetzung}

Was ist das Ziel der Arbeit. Wie soll das Problem gelöst werden?


%% ==============================
\section{Gliederung/Aufbau der Arbeit}
%% ==============================
\label{ch:Einleitung:sec:Gliederung}

Was enthalten die weiteren Kapitel? Wie ist die Arbeit aufgebaut? Welche Methodik wird verfolgt?


%%% Local Variables: 
%%% mode: latex
%%% TeX-master: "thesis"
%%% End: 
