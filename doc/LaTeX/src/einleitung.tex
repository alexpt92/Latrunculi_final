%% Einleitung.tex
%% $Id: einleitung.tex 61 2012-05-03 13:58:03Z bless $
%%

\chapter{Einleitung}
\label{ch:Latrunculi}
%% ==============================
In diesem Kapitel wird die Motivation, die Problemstellung, sowie die Zielsetzung erläutert. Zum Schluss wird die Struktur der Arbeit beschrieben.



%% ==============================
\section{Motivation}
%% ==============================
\label{ch:Einleitung:sec:Motivation}
Da wir uns zum Zeitpunkt der Verfassung dieser Arbeit in einer durch Covid-19 verursachten Pandemie befinden und uns somit Kontaktbeschränkungen im Alltag begleiten, ist die Motivation vor allem den Besuchern des Landesmuseum Birkenfeld ein historisches Spiel online anbieten zu können, da Museumsbesuche nur eingeschränkt möglich sind. Latrunculi wurde implementiert, weil es aufgrund von nicht eindeutig überlieferten Regeln einen gewissen Spielraum in der Umsetzung. Außerdem auch historisch interessante Informationen liefert. Weiterhin bietet das Landesmuseum Birkenfeld Veranstaltungen (\url{https://www.landesmuseum-birkenfeld.de/landesmuseum/erlebnismuseum/}) wie ,,So spielten die Römer'' für Schulklassen an und kann durch eine digitale und spielbare Umsetzung  eines historischen Spiels bei den Besuchern ein größeres Interesse wecken.
%Zugang einfacher!


%%Warum ist das zu bearbeitende Themengebiet spannend und relevant?

%% ==============================
%%\section{Variationen?}
%% ==============================
%%\label{ch:Einleitung:sec:Problemstellung}

%%Welches Problem/welche Probleme können in diesem Themengebiet identifiziert werden?

%% ==============================
\section{Problemstellung}
%% ==============================
\label{ch:Einleitung:sec:Problemstellung}
Latrunculi war historisch gesehen ein beliebtes und weit verbreitetes strategisches Brettspiel der Römer und Griechen gewesen sein. Die römischen Dichter Ovid und Martial haben dieses Spiel bereits erwähnt und beschrieben, dass bei sehr talentierten Spielern häufig Zuschauer anwesend waren. Die Spielsteine wurden als latrones(lat. für Soldat) bezeichnet. Da die Entstehung dieses Spiels so lange zurück liegt, wurden nicht alle Regeln überliefert, sodass verschieden Quellen hinzugezogen werden müssen um den Spielmechanismen zu klären. Das Latrunculi-Spielfeld besteht aus einem Raster senkrechter und waagerechter Reihen. Weiterhin wurden wahrscheinlich die Spielsteine auf den Feldern und nicht auf den Linien platziert und bewegt. Dabei konnte keine fixe Größe des Spielbretts festgestellt werden, sodass verschieden große Spielfelder mit unterschiedlicher Anzahl an Zellen gefunden wurden. Bei Ausgrabungen konnten Latrunculi-Bretter mit beispielsweise 7x7, 8x8, 9x10, 7x10 und 7x6 Feldern geborgen werden. Als Spielbretter dienten hierbei unter anderem Kalksteine oder Ziegelsteine, wie sie in Mainz oder in Hadrianswall in Großbritannien gefunden wurden. Betrachtet man Ovids Aussagen, kann man folgern, dass das Hauptziel darin bestand mit seinen Spielsteinen einen gegnerischen von zwei gegenüberliegenden Seiten zu einzufangen und somit aus dem Spiel zu nehmen. Weiterhin beschreibt er, dass es wichtig sei, seine Steine paarweise zu bewegen, da dadurch verhindert wird, dass diese vom Gegner geschlagen werden können. Für diesen Angriffs- und Verteidigungsmechanismus konnten die Steine geradlinig vorwärts und rückwärts verschoben werden. Allerdings ist auch nicht eindeutig erklärt, wann das Spiel endet oder ob es mit speziellen Figuren funktioniert hat, beispielsweise gibt es Variationen mit einem König mit speziellen Fähigkeiten, ähnlich wie die Dame beim Schach, zwischen den simplen Figuren beziehungsweise Steinen.

%% ==============================
\section{Zielsetzung}
%% ==============================
\label{ch:Einleitung:sec:Zielsetzung}
In dieser Arbeit wird eine nachfolgend erklärte Variation des Spiels Latrunculi mithilfe der Unity Engine für das Landesmuseum Birkenfeld umgesetzt. Dabei wurde eine Künstliche Intelligenz implementiert, sodass es die Möglichkeit gibt, gegen eine KI zu spielen, die zielführende Züge umsetzt. Weiterhin soll das entwickelte Spiel die Möglichkeit bieten, zwei KI's als Simulation gegeneinander antreten zu lassen. Zum Zeitpunkt der Arbeit, befinden wir uns aufgrund von Covid-19 in einer Pandemie, sodass aktuell Kontaktbeschränkungen den Alltag bestimmen, daher soll das Projekt auch die Möglichkeit bieten online abgerufen zu werden. Unity bietet relativ einfache Portierungs-Möglichkeiten, sodass die Entwicklung eines Spiels für unterschiedliche Plattformen erleichtert wird.


%% Was ist das Ziel der Arbeit. Wie soll das Problem gelöst werden?


%% ==============================
\section{Gliederung/Aufbau der Arbeit}
%% ==============================
\label{ch:Einleitung:sec:Gliederung}

%%Was enthalten die weiteren Kapitel? Wie ist die Arbeit aufgebaut? Welche Methodik wird verfolgt?

Die folgenden Kapitel behandeln zuerst die Grundlagen, um die anschließend aufgeführten Abschnitte besser verstehen zu können. Dabei werden die umgesetzten Spielregeln erklärt, sowie die grundsätzliche Definition von Künstlicher Intelligenz, als auch Umsetzungen in bekannten Spielen erwähnt. Außerdem wird die Engine Unity kurz erklärt und der verwendete Algorithmus für die künstliche Intelligenz. Das Dritte Kapitel umfasst sowohl die technischen Anforderungen, als auch nötige Funktionalitäten des Spiels.
Im Anschluss an den dritten Abschnitt werden die nötigen Features und Technologien beschrieben.
Das 5. Kapitel erklärt die Implementierung des Spiels, sowie der KI. Anschließend wird erklärt, wie der Release beim Kunden im Museum unter gegebenen Umständen stattgefunden hat.

%%% Local Variables: 
%%% mode: latex
%%% TeX-master: "thesis"
%%% End: 
